\documentclass{article}
\usepackage[noindent]{ctexcap}
\usepackage{amsmath,amssymb,amsthm,pifont,curves} % math package
\usepackage{multicol} % multicolumns
\usepackage{color}

\title{Note of Copper curve ball}
\author{Gong Yifan}
\date{\today}

\begin{document}
\maketitle
\tableofcontents
本文是对高盛2013/5/22的报告Copper curve ball的笔记。\par
\section{introduction}


\section{中国铜融资循环可能终结}
中国资本控制+中美利差~导致商品质押融资的盛行,以此进行利率套利。
导致近期中国短期外汇贷款的上升(相应的人民币承压),中国外汇管理局(SAFE)宣布了新的管制措施(5/5),将在六月开始实施。\par
我们认为,SAFE的政策将会使中国商品质押在未来1-3个月内终结。但由于政策实施、政策实际影响市场的速度和影响程度的不确定性,CCFDs的彻底终结仍只是一个风险因素\par
我们估计中国约有51万吨的铜(海关仓库内和运输路程中)用于商品质押。如果CCFDs完全结束,铜现货将从正资产(可以用来的赚取利差)成为负资产(持有成本,融资成本高于升水)。
这会导致LME现货承压,我们预计LME将需要额外消化20-25万吨的铜现货,约为全球季度产量的4-5\%。这会导致现货价格下行,更高的期货升水。
目前,LME3-15月升水为1.1\%,持有成本(full carry)为3-3.5\%\par

\section{SAFE的新政策}

\section{典型的CCFD流程}
参与者:\begin{itemize}
    \item[Party\ A] 离岸贸易公司
    \item[Party\ B] 在岸贸易公司
    \item[Party\ C] B的离岸子公司
    \item[Party\ D] 注册在国内的在岸或离岸银行
\end{itemize}
过程:\begin{itemize}
    \item[step 1:] A将海关仓单以价格X卖给B。B获得仓单,A获得D开具的美元信用证。
    \item[step 2:] B把仓单卖给C(B的子公司),C支付美元或离岸人民币。C获得仓单,B获得美元或离岸人民币并通过D换为在岸人民币,获取国内人民币的无风险利率收益。
    \item[step 3:] C把仓单以低于X的价格(一般低10-20\$/t)卖给A。A获得仓单,C获得美元。这里回到了第一步开始的状态。
    \item[step 4:] 重复步骤1-3.在美元信用证期限内(3-12月),一般可以重复10-30次。
\end{itemize}

\section{CCFD中各方的收益}
下面给出一个例子说明在CCFD中每吨铜现货各参与者的收益。\par设上述第四步中循环次数为$c$,
\begin{itemize}
    \item[Party\ A] 第三步中10-20\$*$c$.
    \item[Party\ B] B的成本为美元信用证的利率(约为2.5\%)和仓单循环中支付给A的部分,收益为人民币的利率(约为5\%),以铜价7500\$/t计算,收益为1675-4425\$*$c$.
    \item[Party\ C] B的子公司
    \item[Party\ D] D的成本为央行的融资成本,收益为美元信用证的利率(约为2.5\%),以铜价7500\$/t计算,收益为375-1125\$*$c$.
\end{itemize}

\section{CCFD终结对参与者的影响}
\begin{itemize}
    \item[Party\ A] 没有了每次循环10-20\$/t的收益,A很可能卖出铜现货(期货溢价并不足以抵消仓位租金和利息的成本)。
                    A持有的铜现货可能流入市场,同时A也可能平仓在LME铜期货的空头头寸。
    \item[Party\ B,C] B和C可能会降低它们的美元信用证负债,方法是出售流动资产归还美元信用证负债或借取离岸美元,并把信用证负债展期为离岸的美元负债。
    \item[Party\ D] 为了满足外管局的新监管规则,D将可能通过降低开立信用证的数量和(或)增加外汇净多头头寸.
\end{itemize}

\section{对铜价的影响}

\section{概念}
信用证:
\end{document}
